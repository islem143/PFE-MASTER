\documentclass[12pt]{article}

\usepackage{geometry}
\usepackage[english]{babel}
\usepackage{blindtext}
\usepackage[]{hyperref}
\hypersetup{colorlinks=false,linkcolor=false,pdfborder = {0 0 0}
}
\geometry{top=2cm,bottom=2cm}
\usepackage{setspace}
\usepackage{graphicx}
\graphicspath{{images/}}
\onehalfspacing
\usepackage{pdfpages}
\usepackage[acronym]{glossaries}
\newacronym{ny}{NY}{New York}
\begin{document}

\newpage
\pagebreak
\hspace{0pt}
\vfill
\begin{center}
\section{Introduction}
\end{center}
\vfill
\hspace{0pt}
\pagebreak

\subsection{Introduction}
These days, mobile robots have taken place in many fields like industry automation, planetary exploration, entertainment, and construction ..., for their ability to work in extreme environments with high precision and without fatigue\cite{rubio2019review}.Even so, a robot occasionally needs the support of other robots because it is impossible or difficult for them to perform some tasks on their own. For that, a new field has emmerged to deal with these problems, swarm robotics.

 
Swarm robotics is relatively a new research topic that has gained more attraction in the last few years. It is about  studying how a large number of simple robots (a swarm) can collaborate and work together to achieve predefined objectives and tasks that are often difficult or impossible to do for a single robot.\cite{bayindir2016review}.

One of the main challgenes that swarm robotics researchs face, is pattern formation. Where the agents (robots) try to form diffrent gemoetric shapes like squeres,trinagles and circles in order to perform a specific task.

We can solve this problem using two different aproches. The first one is a Centrelized method where there exist a centel unit  which controles the swarm and give global state access. However,  implementing this apporch  can be coslty and less robust to faillures. The second aproch is a dicentrilzed one, where each robot have uses local commucintion and have access only to his local state.

Our main goal in this thesis is trying to impelment an RL algorithm in a  system consiting of  group of robots  to form some specif paterns using decentriled method.
 


 witout having a centerlized control of the swarm and with limited sense cababilites for each member.








Swarm robotics systems have taken inspiration from how social animals  like (ants, bees, and birds...) behave in groups and that they exhibit some sort of swarm intelligence that researchers identified to be characterized by three properties:\\

\textbf{Robustness}: the ability of the swarm to  still function even with the loss of some members of the group or the faillure of some parts of the system.\\
\textbf{Scalability}: The ability of the system to perform well on smaller or larger group sizes without impacting the performance of the swarm.

\textbf{Flexibility} It is the capability of the swarm to adapt and manage the new changes that occur in the environment \cite{brambilla2013swarm}
  
  
  


 




\bibliography{bib}
\bibliographystyle{ieeetr}



\end{document}